\documentclass{resume}

\usepackage{amssymb}
\usepackage{url}

\renewcommand{\categoryfont}{\sc}

%
% set the space used for category titles here:
% use the same value for oddsidemargin and marginparwidth [the latter 
% 		will be reset to account for marginparsep]
% 
\setlength{\oddsidemargin}{1.1in}
\setlength{\marginparwidth}{1.1in}
% 
% calculate other dimensions [textwidth and evensidemargin] 
% in function of oddsidemargin and marginparwidth: 
% would be nicer to put in the class file...
%
\addtolength{\marginparwidth}{-\marginparsep}
 \setlength{\evensidemargin}{\oddsidemargin}
 \setlength{\textwidth}{\paperwidth}
 \addtolength{\textwidth}{-2in}
 \addtolength{\textwidth}{-2\oddsidemargin}
 \addtolength{\textwidth}{\marginparwidth}
 \addtolength{\textwidth}{\marginparsep}
%
%
\setlength{\topmargin}{-0.5in}
%
%
\renewcommand{\labelcitem}{$\rhd$}
\renewcommand{\labelitemi}{$\cdot$}
\newcommand{\first}{$1^{\mbox{\scriptsize st}}$\ }
\newcommand{\second}{$2^{\mbox{\scriptsize nd}}$\ }
\newcommand{\third}{$3^{\mbox{\scriptsize rd}}$\ }

\author{~~~~~~Michael Andrew Park}
% ------ Address --------------------------------------------------------

\address{
         Computational AeroSciences Branch\\
	 NASA Langley Research Center\\
	 Hampton, VA 23508\\
	 Office +1 (757) 864-6604\\
	 \mbox{\small\tt Mike.Park@NASA.gov}
        }{
	4608 Colonial Avenue\\
	Norfolk, VA 23508\\
	Mobile +1 (757) 535-6675\\
	 \mbox{\small\tt MikePark.rb@Gmail.com}}

\begin{document}
\maketitle

% ------- Education ---------------------------------------------------

\begin{category}{Education}
\citem{Massachusetts Institute of Technology}, Cambridge, MA.\\
Ph.D. in Computational Fluid Dynamics, Department of Aeronautics and Astronautics,  GPA 4.66 of 5.00, September 2008.
\citem{NASA Langley / George Washington University}, Hampton, VA.\\
Joint Institute for the Advancement of Flight Sciences
M.S., Aeronautical Engineering, GPA 3.28 of 4.00, August 2000.
\citem{University of Southern California}, Los Angeles, CA.\\
B.S., Aerospace Engineering, GPA 3.48 of 4.00, May 1998.
\end{category}

% --------- Research ----------------------------------------------------

\begin{category}{Research interests}
\citemnobullet 
Computational Fluid Dynamics (CFD) flow and adjoint
solver development for analysis, solution adaptation, and design;
anisotropic grid adaptation mechanics; parallel computing; software
development processes.
\end{category}

\begin{category}{Skills}
\citemnobullet 
C, Fortran, MATLAB, Ruby, Git, Subversion, GNU Autotools, \LaTeX.
\end{category}

\begin{category}{Work Experience}
\citem{Research Scientist}, Langley Computational AeroSciences Branch;
September 2000--present.
Implemented a three-dimensional output (adjoint) based error
estimation and adaptation scheme in FUN3D.%
\footnote{\url{http://fun3d.larc.nasa.gov}, an unstructured CFD simulation tool with adjoint solver capable of analysis, design, and grid adaptation
across all flow speed regimes} 
Developed a parallel, anisotropic grid adaptation
library called from FUN3D (includes dynamic partition load balancing).
The library is
written in C, developed test-first, and contains a direct link to CAD geometry.
Also developed the cut-cell capability in FUN3D.
Developed automated build and test procedure for FUN3D.
Implemented Message Passing Interface (MPI) communication in FUN3D.
Contributed to FUnit (FORTRAN 95 unit testing framework) Development.
Participated in the AIAA Supersonic Shock-Boundary Layer Interaction and High Lift Prediction Workshops with output-adaptive grid methods.
Applications in sonic boom prediction, engine plumes, and ground vehicles.
Supported numerous FUN3D users in governament, industry, and academia.

\citem{Research Assistant} Langley Multidisciplinary Optimization Branch;
September 1998--August 2000.
Applied the ADIFOR (Automatic Differentiation in FORTRAN) tool to
CFL3D, a FORTRAN, structured grid, Navier-Stokes CFD code to compute
aircraft stability and control derivatives.

\citem{Co-op Flight Test Engineer} 
NASA Dryden Flight Research Center Aerodynamics, Propulsion, and Controls
Branches; September 1995--August 1997. 
Improved angle of attack and
side slip determination during departures for F-18 HARV (High Alpha
Research Vehicle) fight test data. Supported the Linear Aerospike
SR-71 Experiment (LASRE) with supersonic wind tunnel testing and
computational potential flow solutions of possible test configurations
for stability and control data. Programmed a graphical modern control
law design tool in MATLAB. Used the Hyper-X hypersonic free flyer as a
control law design example and explored alternative trajectories for
booster and free flyer separation studies.


\end{category}

% -------- Reference --------------------------------------------

\begin{category}{References Publications} 
\citemnobullet Available on request.
\end{category}

\end{document}
